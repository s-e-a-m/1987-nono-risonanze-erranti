%!TEX TS-program = xelatex
%!TEX encoding = UTF-8 Unicode
%!TEX root = 2022-nono-re.tex
%----------------------------------------------------------------- LANGUAGES ---
\newcommand{\mylanguages}{italian} % in reverse order
%---------------------------------------------------------- TITLE & SUBTITLE ---
\newcommand{\mytitle}{Risonanze Erranti - Luigi Nono}
\newcommand{\mysubtitle}{Appunti AECLE}
%----------------------------------------------------------------- AUTHOR(s) ---
\newcommand{\authorone}{Giuseppe SILVI}
\newcommand{\institutione}{Accademia S. Cecilia di Roma}
\newcommand{\emailone}{grammaton @ me.com}
%-------------------------------------------------------------------------------
% \newcommand{\authortwo}{Wikio Orgopedio}
% \newcommand{\institutiontwo}{Conservatorio S. Cecilia di Roma}
% \newcommand{\emailtwo}{wikio @ orgopedio.com} % duplicate these 3 lines if more
%-------------------------------------------------------------- STYLE GS2020 ---
\AtBeginDocument{\renewcommand\contentsname{\normalfont{MAPPA}}}
%!TEX TS-program = xelatex
%!TEX encoding = UTF-8 Unicode
%!TEX root = 2022-nono-re.tex
%-------------------------------- PACKAGES AND OTHER DOCUMENT CONFIGURATIONS ---
\documentclass[
	a4paper,
	twocolumn,
	twoside,
	%openright
]{article}
\usepackage[
	top=20mm,
	bottom=25mm,
	textwidth=17.2cm,
	columnsep=0.8cm,
	bindingoffset=1cm,
	showframe
]{geometry}
\usepackage[T1]{fontenc}
\usepackage[\mylanguages]{babel}
\usepackage{csquotes}
%\usepackage{parskip}
\usepackage[style=authoryear-ibid,backend=biber]{biblatex}
\bibliography{includes/bibliography.bib}
\usepackage{dblfloatfix}
\usepackage{subfigure}
\usepackage[subfigure]{tocloft}
\advance\cftsecnumwidth 0.5em\relax
\advance\cftsubsecindent 0.5em\relax
\advance\cftsubsecnumwidth 0.5em\relax
\usepackage{graphicx}
\usepackage{wrapfig}
% \usepackage{epstopdf}
% \epstopdfsetup{update}
\usepackage[usenames]{color}
\usepackage{xcolor}
\usepackage{tikz}
\usetikzlibrary{shapes,
                through,
								calc,
								intersections,
								backgrounds,
                positioning}
\usepackage{tkz-euclide}
\usepackage{amssymb}
\usepackage[
  colorlinks=true,
  linkcolor=black,
	anchorcolor=black,
	citecolor=black,
	filecolor=black,
	menucolor=black,
	runcolor=black,
	urlcolor=black
	]{hyperref}
\usepackage{Alegreya}
\linespread{1.05}
\usepackage{
	fontspec,
	xltxtra,
	xunicode
	}
\usepackage{
	xfrac,
	unicode-math
	}

\defaultfontfeatures{Mapping=tex-text}
\setmonofont[
	Scale=MatchLowercase
	]{Andale Mono}
\setmathfont[
	Scale=MatchLowercase,
	Scale=1
	]{Libertinus Math}

\usepackage{microtype}

\usepackage[
	hang,
	small,
	labelfont=bf,
	up,
	textfont=it,
	up
	]{caption}
\usepackage{paralist} % For compact item lists
\usepackage{etoolbox} % Some tools: used for quote environment
\AtBeginEnvironment{quote}{\small}
\usepackage{titling} % Customizing the title section
\usepackage{booktabs} % Horizontal rules in tables
\usepackage{enumitem} % Customized lists
\setlist[itemize]{noitemsep} % Make itemize lists more compact
\usepackage{abstract} % Allows abstract customization
\renewcommand{\abstractnamefont}{\normalfont\bfseries} % Set the "Abstract" text to bold
\renewcommand{\abstracttextfont}{\normalfont\small\itshape} % Set the abstract itself to small italic text
\usepackage{titlesec} % Allows customization of titles
\renewcommand\thesection{\Roman{section}} % Roman numerals for the sections
\renewcommand\thesubsection{\Roman{subsection}} % roman numerals for subsections
\titleformat{\section}[block]{\Large}{\thesection.}{1em}{} % Change the look of the section titles
\titleformat{\subsection}[block]{\large}{\thesubsection.}{1em}{} % Change the look of the section titles
%------------------------------------------------------------- TITLE SECTION ---
\setlength{\droptitle}{-4\baselineskip} % Move the title up
\pretitle{\begin{center}\huge\bfseries} % Article title formatting
\posttitle{\end{center}} % Article title closing formatting
\title{\mytitle \\[0.1cm] \large{\emph{\mysubtitle}}} % Article title
\author{%
\textsc{\authorone}\\%
\normalsize \institutione \\ %
\normalsize \emailone %
% activate
% \and % duplicate these 4 lines if more
% \textsc{\authortwo} \\%
% \normalsize \institutiontwo \\ %
% \normalsize \emailtwo %
}
\date{} % Leave empty to omit a date

\usepackage{fancyhdr} % Headers and footers
\pagestyle{fancy} % All pages have headers and footers
\fancyhead{} % Blank out the default header
\fancyfoot{} % Blank out the default footer
\fancyhead[C]{\small Giuseppe SILVI • Canto alla durata} % Custom header text
\fancyfoot[RO]{\small \today~ • w: \input{includes/words.txt} • c: \input{includes/char.txt} • p:~\thepage} % Custom footer text
\fancyfoot[LE]{\small p:~\thepage~ • c: \input{includes/char.txt} • w: \input{includes/words.txt} • \today} % Custom footer text
%-------------------------------------------------------------------------------
%-------------------------------------------------------------------------------
%	LISTINGS
%-------------------------------------------------------------------------------
%-------------------------------------------------------------------------------
\usepackage{listings}
% lstlistings setup
\definecolor{gsbg}{rgb}{0.98,0.98,0.98}

\lstset{%
  aboveskip=10pt,
	belowskip=5pt,
  language=C++,
  numbers=none,%left,%none,
  tabsize=4,
  %frame=single,
  breaklines=true,
  numberstyle=\tiny\ttfamily,
  backgroundcolor=\color{gsbg},
  basicstyle=\footnotesize\ttfamily,
  %commentstyle=\slshape\color{mylstcmt}, %\itshape,
  %frameround=tttt,
  columns=flexible, %fixed,
  showstringspaces=false,
  emptylines=2,
  inputencoding=utf8,
  extendedchars=true,
  literate=	{á}{{\'a}}1
			{à}{{\`a}}1
			{ä}{{\"a}}1
			{â}{{\^a}}1
			{é}{{\'e}}1
			{è}{{\`e}}1
			{ë}{{\"e}}1
			{ê}{{\^e}}1
			{ï}{{\"i}}1
			{î}{{\^i}}1
			{ö}{{\"o}}1
			{ô}{{\^o}}1
			{è}{{\`e}}1
			{ù}{{\`u}}1
			{û}{{\^u}}1
			{ç}{{\c{c}}}1
			{Ç}{{\c{C}}}1,
  emph={component, declare, environment, import, library, process},
  emph={[2]ffunction, fconstant, fvariable},
  emph={[3]button, checkbox, vslider, hslider, nentry, vgroup, hgroup, tgroup, vbargraph, hbargraph, attach},
  %emphstyle=\color{yotxt}, %\underline, %\bfseries,
  %morecomment=[s][\color{mylstdoc}]{<mdoc>}{</mdoc>},
  rulecolor=\color{black}
}

\usepackage[framemethod=tikz]{mdframed} % Allows defining custom boxed/framed environments

%-------------------------------------------------------------------------------
%--------------------------------------------------- INFORMATION ENVIRONMENT ---
%-------------------------------------------------------------------------------

% Usage:
% \begin{info}[optional title, defaults to "Info:"]
% 	contents
% 	\end{info}

\mdfdefinestyle{info}{%
	topline=false, bottomline=false,
	leftline=false, rightline=false,
	nobreak,
	singleextra={%
		\fill[black](P-|O)circle[radius=0.4em];
		\node at(P-|O){\color{white}\scriptsize\bf i};
		\draw[very thick](P-|O)++(0,-0.8em)--(O);%--(O-|P);
	}
}

% Define a custom environment for information
\newenvironment{info}[1][Info:]{ % Set the default title to "Info:"
	\medskip
	\begin{mdframed}[style=info]
		\footnotesize\noindent{\textbf{#1}}
}{
	\end{mdframed}
}

%-------------------------------------------------------------------------------
%----------------------------------------------------- BIOGRAFIA ENVIRONMENT ---
%-------------------------------------------------------------------------------

% Usage:
% \begin{bio}[optional title, defaults to "Info:"]
% 	contents
% 	\end{bio}

\mdfdefinestyle{bio}{%
	topline=false, bottomline=false,
	leftline=false, rightline=false,
	nobreak,
	singleextra={%
		\fill[black](P-|O)circle[radius=0.4em];
		\node at(P-|O){\color{white}\scriptsize\bf b};
		\draw[very thick](P-|O)++(0,-0.8em)--(O);%--(O-|P);
	}
}

% Define a custom environment for information
\newenvironment{bio}[1][Biografia:]{ % Set the default title to "Info:"
	\medskip
	\begin{mdframed}[style=bio]
		\noindent{\textbf{#1}}
}{
	\end{mdframed}
}

%-------------------------------------------------------------------------------
%------------------------------------------------------- WARNING ENVIRONMENT ---
%-------------------------------------------------------------------------------

% Usage:
% \begin{warn}[optional title, defaults to "Warning:"]
%	Contents
% \end{warn}

\mdfdefinestyle{warning}{
	topline=false, bottomline=false,
	leftline=false, rightline=false,
	nobreak,
	singleextra={%
		\draw(P-|O)++(-0.5em,0)node(tmp1){};
		\draw(P-|O)++(0.5em,0)node(tmp2){};
		\fill[black,rotate around={45:(P-|O)}](tmp1)rectangle(tmp2);
		\node at(P-|O){\color{white}\scriptsize\bf !};
		\draw[very thick](P-|O)++(0,-1em)--(O);%--(O-|P);
	}
}

% Define a custom environment for warning text
\newenvironment{warn}[1][Warning:]{ % Set the default warning to "Warning:"
	\medskip
	\begin{mdframed}[style=warning]
		\noindent{\textbf{#1}}
}{
	\end{mdframed}
}

%-------------------------------------------------------------------- ABSTRACT -
\renewcommand{\maketitlehookd}{%
\begin{abstract}
\noindent\input{includes/abstract.txt}
\end{abstract}
}

\newcommand{\canto}{\emph{canto}}

%------------------------------------------------------------ BEGIN DOCUMENT ---
\begin{document}
\maketitle
\thispagestyle{empty}
%-------------------------------------------------------------------- ABSTRACT -
% The abstract is an external txt file inside the includes folder
%-------------------------------------------------------------------------------
%-------------------------------------------------------------------------------
%------------------------------------------------------------------- SECTION ---
%-------------------------------------------------------------------------------
\section*{NOTE DI PROGRAMMA}

Il testo si apre con “\emph{… la durata induce alla poesia. | Voglio interrogarmi con
un canto, | voglio ricordare con un canto, | dire e affidare a un canto | cos'è
la durata.}”

\newpage
%-------------------------------------------------------------------------------
%-------------------------------------------------------------------------------
%-------------------------------------------------------------------------------
%La mappatura della membrana del timpano.
%https://en.wikipedia.org/wiki/Lune_of_Hippocrates
\begin{figure}[h]
\centering
\resizebox{0.47\textwidth}{!}{%
\begin{tikzpicture}%[scale=0.93]
    \path[draw] (0,4) coordinate (Q)
                (4,0) coordinate [label=right:$B$] (B)
             -- (-4,0) coordinate [label=left:$A$] (A)
             -- (0,-4) coordinate (K)
            ($(A)!0.5!(K)$) coordinate[label=below:$D$](D);
    \foreach \point in {A,B,K,D} \fill [black] (\point) circle (2pt);
    \draw [color=black] circle(4cm);
    \scoped[on background layer]
    {
    \draw[name path=arc1,color=purple]
         let \p1 = ($ (B) - (K) $),
             \n1 = {veclen(\x1,\y1)} in
         (B) arc (-45:-135:\n1);
    \draw[name path=arc2,color=green,rotate=90]
         let \p1 = ($ (B) - (K) $),
             \n1 = {veclen(\x1,\y1)} in
         (Q) arc (-45:-135:\n1);
    \path [name intersections={of=arc1 and arc2,by=C}];
    \fill [black] (C) circle (2pt);
    \node[label=below:$C$] at (C) {};
    }
\end{tikzpicture}
}
\caption{Due lune, Ippocrate su pelle. 2022.}
\label{tikz:duelune}
\end{figure}

\vfill
\tableofcontents
\clearpage
%-------------------------------------------------------------------------------
%-------------------------------------------------------------------------------
%-------------------------------------------------------------------------------
% \begin{figure*}[t]
% \centering
% \includegraphics[width=.97\textwidth]{img/messainscena.jpeg}
% \caption{Luogo e messa in scena}
% \label{luogo}
% \end{figure*}
%-------------------------------------------------------------------------------
%------------------------------------------------------------------- SECTION ---
%-------------------------------------------------------------------------------
\section{LUOGO SONORO}

\begin{warn}[ellisse]
s. f. [dal lat. scient. ellipsis, e questo dal gr. ἔλλειψις «mancanza»]. – In
geometria, curva piana chiusa appartenente alla famiglia delle coniche [\ldots]
In astronomia, è un’ellisse l’orbita descritta da un pianeta intorno al Sole
(che ne occupa un fuoco), quella di un satellite intorno a un pianeta, ecc. \\
- \url{treccani.it}.
\end{warn}

\begin{figure}[t]
\centering
\includegraphics[width=.47\textwidth]{img/image1.png}
\caption{Luogo e messa in scena.}
\label{luogo}
\end{figure}

Il complesso risultante è quello di un sistema ottofonico, otto elementi che
costruiscono l'organismo sensibile e sensiente, voce e ascolto, suono quindi
luogo. Otto voci, rivolte verso il centro su cui sovrasta un diffusore
omnidirezionale a banda larga \emph{S.T.ON3L}, come sul vertice di una piramide
a base ottogonale inclinata, dal più basso al più alto: $ P-III-S-II-V $ scendendo
$ I-E-II-[P] $. [fig. \ref{luogo}] Il luogo acustico che si crea
in questa configurazione è interno al suono interpretato, all'attività musicale,
luogo geometrico, ritmico, poetico. Il canto di quattro volte infinite voci in
uno spazio in risonanza. Non una proiezione esterna quindi, come accade nella
tradizionale diffusione frontale. Non c'è un fronte. Nella messa in scena ideale
il pubblico dovrebbe circondare il luogo sonoro, meglio, attraversarlo. Nel caso
in cui questo non fosse possibile, il luogo può essere arbitrariamente osservato
dall'esterno, in maniera circolare o da una prospettiva; tra le prospettive
esterne la più indicata si ottiene posizionando il pubblico rivolto verso la
terna $ P-III-S $.

\begin{quote}
Ogni volto fotografato\\
è un'immagine bellica,\\
il punto di tangenza\\
tra l'aereo nemico e la nave\\
nell'attimo che precede l'esplosione.\\
Fermo nell'istantanea,\\
nel contatto flagrante tra due sguardi\\
immolato, ripreso\\
mentre le fiamme covano già\\
nella fusoliera crescendo\\
dentro i suoi tratti, vive\\
soltatno il tempo necessario\\
a compiere la missione del ricordo. \footcite{Magrelli2018}
\end{quote}

\begin{equation}
\left\{ \begin{aligned}
x & = \left(r + \cos\frac{\theta}{2}\sin v - \sin\frac{\theta}{2}\sin 2v\right) \cos \theta\\
y & = \left(r + \cos\frac{\theta}{2}\sin v - \sin\frac{\theta}{2}\sin 2v\right) \sin \theta\\
z & = \sin\frac{\theta}{2}\sin v + \cos\frac{\theta}{2}\sin 2v
\end{aligned} \right.
\end{equation}

%------------------------------------------------------
%------------------------- larghezza massima del codice
\begin{lstlisting}
//-----------------------------------------------------
//------- FIGURE OF 8 IMMERSION OF THE KLEIN BOTTLE ---
//-----------------------------------------------------
import("stdfaust.lib");
//------------------------------ DEGREES TO RADIANS ---
d2r = *(ma.PI/180);
//--------------------------------------- VARIABLES ---
a = hslider("angle", 0,0,360,0.1);
v = hslider("v", 0,0,360,0.1);
//-------------------------------------------- MATH ---
x(a,r,v) = (r + (cos(a/2) * sin(v)) -
           (sin(a/2) * sin(2*v))) * cos(a) :
           si.smoo;
y(a,r,v) = (r + (cos(a/2) * sin(v)) -
           (sin(a/2) * sin(2*v))) * sin(a) :
           si.smoo;
z(a,v) = (sin(a/2) * sin(v)) +
         (cos(a/2) * sin(2*v)) :
         si.smoo;
fig8immersion(a,r,v) = _,
                       *(x(d2r(a),2,d2r(v))),
                       *(y(d2r(a),2,d2r(v))),
                       *(z(d2r(a),d2r(v)));
//----------------------------- BAND-FILTERED NOISE ---
bandnoise(f) = no.noise :
               fi.highpass(64,f) :
               fi.lowpass(64,f);
//------------------------- AMBISONIC B to A-FORMAT ---
bamodule(W,X,Y,Z) = LFU,RFD,RBU,LBD
  with{
	  LFU = 0.5 * (W + X + Y + Z);
	  RFD = 0.5 * (W + X - Y - Z);
	  RBU = 0.5 * (W - X - Y + Z);
    LBD = 0.5 * (W - X + Y - Z);
};
//-------------------------------------------- SPAT ---
spat(n,a)	= par(z,n,*(scaler(z,n,a) : si.smoo)),
            par(z,n,*(scaler(z,n,a) : si.smoo)),
            par(z,n,*(scaler(z,n,a) : si.smoo)),
            par(z,n,*(scaler(z,n,a) : si.smoo))
            with {
              scaler(z,n,a) = sqrt(max(0.0, 1.0 -
                abs(fmod(a+0.5+float(n-z)/n, 1.0) -
                0.5) * n));
};
stspat(n,a) = spat(n,a) : ro.interleave(n,n);
//----------------------------------------- EXAMPLE ---
process = 0, bandnoise(1000), bandnoise(1002),0 <:
          par(i,4, fig8immersion((i/4)*2*ma.PI,2,v) :
          bamodule) :
          stspat(4,a/360);
\end{lstlisting}

\end{document}

%%%%%%%%%%%%%%%%%%%%%%%%%%%%%%%%%%%%%%%%%%%%%%%%%%%%%%%%%%%%%%%%%%%%%%%%%%%%%%%%
% 2020 GIUSEPPE SILVI ARTICLE TEMPLATE BASED ON
%%%%%%%%%%%%%%%%%%%%%%%%%%%%%%%%%%%%%%%%%%%%%%%%%%%%%%%%%%%%%%%%%%%%%%%%%%%%%%%%
% Journal Article
% LaTeX Template
% Version 1.4 (15/5/16)
% This template has been downloaded from:
% http://www.LaTeXTemplates.com
% Original author:
% Frits Wenneker (http://www.howtotex.com) with extensive modifications by
% Vel (vel@LaTeXTemplates.com)
% License:
% CC BY-NC-SA 3.0 (http://creativecommons.org/licenses/by-nc-sa/3.0/)
%%%%%%%%%%%%%%%%%%%%%%%%%%%%%%%%%%%%%%%%%%%%%%%%%%%%%%%%%%%%%%%%%%%%%%%%%%%%%%%%
