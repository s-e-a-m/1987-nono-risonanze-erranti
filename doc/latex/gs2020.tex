%!TEX TS-program = xelatex
%!TEX encoding = UTF-8 Unicode
%!TEX root = 2022-nono-re.tex
%-------------------------------- PACKAGES AND OTHER DOCUMENT CONFIGURATIONS ---
\documentclass[
	a4paper,
	twocolumn,
	twoside,
	%openright
]{article}
\usepackage[
	top=20mm,
	bottom=25mm,
	textwidth=17.2cm,
	columnsep=0.8cm,
	bindingoffset=1cm,
	showframe
]{geometry}
\usepackage[T1]{fontenc}
\usepackage[\mylanguages]{babel}
\usepackage{csquotes}
%\usepackage{parskip}
\usepackage[style=authoryear-ibid,backend=biber]{biblatex}
\bibliography{includes/bibliography.bib}
\usepackage{dblfloatfix}
\usepackage{subfigure}
\usepackage[subfigure]{tocloft}
\advance\cftsecnumwidth 0.5em\relax
\advance\cftsubsecindent 0.5em\relax
\advance\cftsubsecnumwidth 0.5em\relax
\usepackage{graphicx}
\usepackage{wrapfig}
% \usepackage{epstopdf}
% \epstopdfsetup{update}
\usepackage[usenames]{color}
\usepackage{xcolor}
\usepackage{tikz}
\usetikzlibrary{shapes,
                through,
								calc,
								intersections,
								backgrounds,
                positioning}
\usepackage{tkz-euclide}
\usepackage{amssymb}
\usepackage[
  colorlinks=true,
  linkcolor=black,
	anchorcolor=black,
	citecolor=black,
	filecolor=black,
	menucolor=black,
	runcolor=black,
	urlcolor=black
	]{hyperref}
\usepackage{Alegreya}
\linespread{1.05}
\usepackage{
	fontspec,
	xltxtra,
	xunicode
	}
\usepackage{
	xfrac,
	unicode-math
	}

\defaultfontfeatures{Mapping=tex-text}
\setmonofont[
	Scale=MatchLowercase
	]{Andale Mono}
\setmathfont[
	Scale=MatchLowercase,
	Scale=1
	]{Libertinus Math}

\usepackage{microtype}

\usepackage[
	hang,
	small,
	labelfont=bf,
	up,
	textfont=it,
	up
	]{caption}
\usepackage{paralist} % For compact item lists
\usepackage{etoolbox} % Some tools: used for quote environment
\AtBeginEnvironment{quote}{\small}
\usepackage{titling} % Customizing the title section
\usepackage{booktabs} % Horizontal rules in tables
\usepackage{enumitem} % Customized lists
\setlist[itemize]{noitemsep} % Make itemize lists more compact
\usepackage{abstract} % Allows abstract customization
\renewcommand{\abstractnamefont}{\normalfont\bfseries} % Set the "Abstract" text to bold
\renewcommand{\abstracttextfont}{\normalfont\small\itshape} % Set the abstract itself to small italic text
\usepackage{titlesec} % Allows customization of titles
\renewcommand\thesection{\Roman{section}} % Roman numerals for the sections
\renewcommand\thesubsection{\Roman{subsection}} % roman numerals for subsections
\titleformat{\section}[block]{\Large}{\thesection.}{1em}{} % Change the look of the section titles
\titleformat{\subsection}[block]{\large}{\thesubsection.}{1em}{} % Change the look of the section titles
%------------------------------------------------------------- TITLE SECTION ---
\setlength{\droptitle}{-4\baselineskip} % Move the title up
\pretitle{\begin{center}\huge\bfseries} % Article title formatting
\posttitle{\end{center}} % Article title closing formatting
\title{\mytitle \\[0.1cm] \large{\emph{\mysubtitle}}} % Article title
\author{%
\textsc{\authorone}\\%
\normalsize \institutione \\ %
\normalsize \emailone %
% activate
% \and % duplicate these 4 lines if more
% \textsc{\authortwo} \\%
% \normalsize \institutiontwo \\ %
% \normalsize \emailtwo %
}
\date{} % Leave empty to omit a date

\usepackage{fancyhdr} % Headers and footers
\pagestyle{fancy} % All pages have headers and footers
\fancyhead{} % Blank out the default header
\fancyfoot{} % Blank out the default footer
\fancyhead[C]{\small Giuseppe SILVI • Canto alla durata} % Custom header text
\fancyfoot[RO]{\small \today~ • w: \input{includes/words.txt} • c: \input{includes/char.txt} • p:~\thepage} % Custom footer text
\fancyfoot[LE]{\small p:~\thepage~ • c: \input{includes/char.txt} • w: \input{includes/words.txt} • \today} % Custom footer text
%-------------------------------------------------------------------------------
%-------------------------------------------------------------------------------
%	LISTINGS
%-------------------------------------------------------------------------------
%-------------------------------------------------------------------------------
\usepackage{listings}
% lstlistings setup
\definecolor{gsbg}{rgb}{0.98,0.98,0.98}

\lstset{%
  aboveskip=10pt,
	belowskip=5pt,
  language=C++,
  numbers=none,%left,%none,
  tabsize=4,
  %frame=single,
  breaklines=true,
  numberstyle=\tiny\ttfamily,
  backgroundcolor=\color{gsbg},
  basicstyle=\footnotesize\ttfamily,
  %commentstyle=\slshape\color{mylstcmt}, %\itshape,
  %frameround=tttt,
  columns=flexible, %fixed,
  showstringspaces=false,
  emptylines=2,
  inputencoding=utf8,
  extendedchars=true,
  literate=	{á}{{\'a}}1
			{à}{{\`a}}1
			{ä}{{\"a}}1
			{â}{{\^a}}1
			{é}{{\'e}}1
			{è}{{\`e}}1
			{ë}{{\"e}}1
			{ê}{{\^e}}1
			{ï}{{\"i}}1
			{î}{{\^i}}1
			{ö}{{\"o}}1
			{ô}{{\^o}}1
			{è}{{\`e}}1
			{ù}{{\`u}}1
			{û}{{\^u}}1
			{ç}{{\c{c}}}1
			{Ç}{{\c{C}}}1,
  emph={component, declare, environment, import, library, process},
  emph={[2]ffunction, fconstant, fvariable},
  emph={[3]button, checkbox, vslider, hslider, nentry, vgroup, hgroup, tgroup, vbargraph, hbargraph, attach},
  %emphstyle=\color{yotxt}, %\underline, %\bfseries,
  %morecomment=[s][\color{mylstdoc}]{<mdoc>}{</mdoc>},
  rulecolor=\color{black}
}

\usepackage[framemethod=tikz]{mdframed} % Allows defining custom boxed/framed environments

%-------------------------------------------------------------------------------
%--------------------------------------------------- INFORMATION ENVIRONMENT ---
%-------------------------------------------------------------------------------

% Usage:
% \begin{info}[optional title, defaults to "Info:"]
% 	contents
% 	\end{info}

\mdfdefinestyle{info}{%
	topline=false, bottomline=false,
	leftline=false, rightline=false,
	nobreak,
	singleextra={%
		\fill[black](P-|O)circle[radius=0.4em];
		\node at(P-|O){\color{white}\scriptsize\bf i};
		\draw[very thick](P-|O)++(0,-0.8em)--(O);%--(O-|P);
	}
}

% Define a custom environment for information
\newenvironment{info}[1][Info:]{ % Set the default title to "Info:"
	\medskip
	\begin{mdframed}[style=info]
		\footnotesize\noindent{\textbf{#1}}
}{
	\end{mdframed}
}

%-------------------------------------------------------------------------------
%----------------------------------------------------- BIOGRAFIA ENVIRONMENT ---
%-------------------------------------------------------------------------------

% Usage:
% \begin{bio}[optional title, defaults to "Info:"]
% 	contents
% 	\end{bio}

\mdfdefinestyle{bio}{%
	topline=false, bottomline=false,
	leftline=false, rightline=false,
	nobreak,
	singleextra={%
		\fill[black](P-|O)circle[radius=0.4em];
		\node at(P-|O){\color{white}\scriptsize\bf b};
		\draw[very thick](P-|O)++(0,-0.8em)--(O);%--(O-|P);
	}
}

% Define a custom environment for information
\newenvironment{bio}[1][Biografia:]{ % Set the default title to "Info:"
	\medskip
	\begin{mdframed}[style=bio]
		\noindent{\textbf{#1}}
}{
	\end{mdframed}
}

%-------------------------------------------------------------------------------
%------------------------------------------------------- WARNING ENVIRONMENT ---
%-------------------------------------------------------------------------------

% Usage:
% \begin{warn}[optional title, defaults to "Warning:"]
%	Contents
% \end{warn}

\mdfdefinestyle{warning}{
	topline=false, bottomline=false,
	leftline=false, rightline=false,
	nobreak,
	singleextra={%
		\draw(P-|O)++(-0.5em,0)node(tmp1){};
		\draw(P-|O)++(0.5em,0)node(tmp2){};
		\fill[black,rotate around={45:(P-|O)}](tmp1)rectangle(tmp2);
		\node at(P-|O){\color{white}\scriptsize\bf !};
		\draw[very thick](P-|O)++(0,-1em)--(O);%--(O-|P);
	}
}

% Define a custom environment for warning text
\newenvironment{warn}[1][Warning:]{ % Set the default warning to "Warning:"
	\medskip
	\begin{mdframed}[style=warning]
		\noindent{\textbf{#1}}
}{
	\end{mdframed}
}

%-------------------------------------------------------------------- ABSTRACT -
\renewcommand{\maketitlehookd}{%
\begin{abstract}
\noindent\input{includes/abstract.txt}
\end{abstract}
}

\newcommand{\canto}{\emph{canto}}
